\documentclass[stu,12pt]{apa7}
  \usepackage{times}               % Times New Roman Font Face
  \usepackage[american]{babel}     % Localization
  \usepackage[utf8]{inputenc}      % Input Encoding
  \usepackage{hyperref}            % Hyperlinks
  \usepackage{enumitem}            % Additional Enumeration Environment Settings
  \usepackage{geometry}            % Page Layout
  \usepackage{soul}                % Text Highlighting
  \usepackage{graphicx}            % Images
  \usepackage{csquotes}            % Quoting Environment
  \usepackage{bookmark}            % Required by `csquotes'
  \usepackage{mdframed}            % Colorful Tex-Box Environment
  \usepackage[toc]{appendix}       % Appendix
  \usepackage{fancyhdr}            % Headings and Footers
  \usepackage[%
    style=apa,%
    sortcites=true,%
    sorting=nyt%
  ]{biblatex}
  \usepackage{xcolor}

  % Bibliography Setup
  %% Language Mappings
  \DeclareLanguageMapping{english}{english-apa}
  \DeclareLanguageMapping{american}{american-apa}
  %% Bibliography File Path
  \addbibresource{main.bib}
  %% Categories for Specified Bibliography Items
  %%% Category for sources not referenced in-text
  \DeclareBibliographyCategory{consulted}
  \addtocategory{consulted}{noauthor_business_nodate}
  \addtocategory{consulted}{noauthor_college_nodate}
  \addtocategory{consulted}{noauthor_communication_2013}


  % Hyperlink Setup
  \hypersetup{
    colorlinks = true,
    urlcolor = blue,
    linkcolor = blue,
    citecolor = blue
  }

  % Page and Text Layout
  \geometry{%
    a4paper,%
    top=1in,%
    bottom=1in,%
    left=1in,%
    right=1in%
  }
  \setlength{\headheight}{15pt}


  % Title Page
  \title{%
    Hindsight Is Always 20/20
  }
  \shorttitle{Module 5 Essay Assignment}
  \author{Ashton Hellwig}
  \authorsaffiliations{Department of Mathematics, Front Range Community College}
  \course{COM125: Interpersonal Communication}
  \professor{Richard Thomas}
  \duedate{December 12, 2020 23:59:59 MDT}
  \date{\today}
  \lhead{COM125CG1-M5E}
  \abstract{%
    \textbf{Overview}\\%
    The end of any course ought to bring with it a sense of having had a
      cumulative experience, the idea that your work in the course has produced
      something you can take with you, that you have engaged in a process of
      thought and activities that adds up to something, and that you have
      learned something about interpersonal communication and yourself as a
      communicator.\\%

    In this course, experiencing the sense of having learned something about
      yourself as a communicator is especially important because you have been
      evaluating and practicing new skills. For your final essay, write a
      retrospective analysis about what you have learned and how you have
      learned it.\\%

    You have several resources for this endeavor, including, but not limited
      to, the course syllabus, the course readings, your discussion board posts,
      your assignments, even the grading feedback you received from the
      instructor.\\%

    Your essay will be first person in voice and self-reflective in tone. It
      will use your own previous assignments as evidence in support of your
      final analysis of the semester's work.\\%

    You should spend approximately 6.5 hours on this assignment.%
  }


\begin{document}
  % Title Page
  \maketitle


  \section*{Instructions}
    \begin{enumerate}
      \item Consider the following questions. Then, compose a reflective paper
        of 800--1,000 words based on the following:
        \begin{itemize}
          \item Reread the course syllabus. The syllabus makes specific
            references to course objectives. Ask yourself if you met those
            objectives, or if you wish you had focused more on any one of them.
          \item Reread your discussion board posts and comments. Are any
            patterns revealed in terms of your responses, handling of the
            readings or videos, and/or connections to the assignments? Can
            you describe those patterns?
          \item Reread your assignments and your grading feedback. Were some
            assignments more difficult, more rewarding, or more reflective of
            your interpersonal communication style?
          \item Consider the interpersonal communication skills we studied in
            this course. Choose at least one communication skill or topic from
            each module to answer this question: do you see patterns in terms
            of your responses to those skills or your ability to use them
            effectively?
          \item As related to interpersonal communication, has this course
            enabled you to better become the ``real you'', as Flint encourages
            Sam in the movie clip? Outside of our class, who has most helped
            you become a better interpersonal communicator?
        \end{itemize}
      \item Refrain from responding to the aforementioned questions in
        chronological order, which may result in a final essay that is merely
        a rote reply to each question. Instead, enable the aforementioned
        questions to provide a framework as you begin working on the essay.
      \item Reference your own previous work in the class (Main Discussion
        posts and/or Essays) to support some or all of your responses to the
        aforementioned questions.  Use either APA or MLA Style to format your
        references (minimum of three references).
    \end{enumerate}


  \section{Introduction}
    Placeholder.


  % Bibliography
  %% Works Cited
  % \newpage
  % \printbibliography[%
  %   title={References},%
  %   heading={bibintoc},%
  %   notcategory={consulted}%
  % ]


  %% Works Consulted
  \newpage
  \nocite{*}
  \printbibliography[%
    title={Additional References},%
    heading={bibintoc},%
    category={consulted}%
  ]
\end{document}
