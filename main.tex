\documentclass[stu,12pt]{apa7}
  \usepackage{times}               % Times New Roman Font Face
  \usepackage[american]{babel}     % Localization
  \usepackage[utf8]{inputenc}      % Input Encoding
  \usepackage{hyperref}            % Hyperlinks
  \usepackage{enumitem}            % Additional Enumeration Environment Settings
  \usepackage{geometry}            % Page Layout
  \usepackage{soul}                % Text Highlighting
  \usepackage{graphicx}            % Images
  \usepackage{csquotes}            % Quoting Environment
  \usepackage{bookmark}            % Required by `csquotes'
  \usepackage{mdframed}            % Colorful Tex-Box Environment
  \usepackage[toc]{appendix}       % Appendix
  \usepackage{fancyhdr}            % Headings and Footers
  \usepackage[%
    style=apa,%
    sortcites=true,%
    sorting=nyt%
  ]{biblatex}
  \usepackage{xcolor}

  % Bibliography Setup
  %% Language Mappings
  \DeclareLanguageMapping{english}{english-apa}
  \DeclareLanguageMapping{american}{american-apa}
  %% Bibliography File Path
  \addbibresource{main.bib}
  %% Categories for Specified Bibliography Items
  %%% Category for sources not referenced in-text
  \DeclareBibliographyCategory{consulted}
  \addtocategory{consulted}{noauthor_business_nodate}
  \addtocategory{consulted}{noauthor_college_nodate}
  \addtocategory{consulted}{noauthor_communication_2013}


  % Hyperlink Setup
  \hypersetup{
    colorlinks = true,
    urlcolor = blue,
    linkcolor = blue,
    citecolor = blue
  }

  % Page and Text Layout
  \geometry{%
    a4paper,%
    top=1in,%
    bottom=1in,%
    left=1in,%
    right=1in%
  }
  \setlength{\headheight}{15pt}


  % Title Page
  \title{%
    Hindsight Is Always 20/20
  }
  \shorttitle{Module 5 Essay Assignment}
  \author{Ashton Hellwig}
  \authorsaffiliations{Department of Mathematics, Front Range Community College}
  \course{COM125: Interpersonal Communication}
  \professor{Richard Thomas}
  \duedate{December 12, 2020 23:59:59 MDT}
  \date{\today}
  \lhead{COM125CG1-M5E}
  \abstract{%
    \textbf{Overview}\\%
    The end of any course ought to bring with it a sense of having had a
      cumulative experience, the idea that your work in the course has produced
      something you can take with you, that you have engaged in a process of
      thought and activities that adds up to something, and that you have
      learned something about interpersonal communication and yourself as a
      communicator.\\%

    In this course, experiencing the sense of having learned something about
      yourself as a communicator is especially important because you have been
      evaluating and practicing new skills. For your final essay, write a
      retrospective analysis about what you have learned and how you have
      learned it.\\%

    You have several resources for this endeavor, including, but not limited
      to, the course syllabus, the course readings, your discussion board posts,
      your assignments, even the grading feedback you received from the
      instructor.\\%

    Your essay will be first person in voice and self-reflective in tone. It
      will use your own previous assignments as evidence in support of your
      final analysis of the semester's work.\\%

    You should spend approximately 6.5 hours on this assignment.%
  }


\begin{document}
  % Title Page
  \maketitle


  \section*{Instructions}
    \begin{enumerate}
      \item Consider the following questions. Then, compose a reflective paper
        of 800--1,000 words based on the following:
        \begin{itemize}
          \item Reread the course syllabus. The syllabus makes specific
            references to course objectives. Ask yourself if you met those
            objectives, or if you wish you had focused more on any one of them.
          \item Reread your discussion board posts and comments. Are any
            patterns revealed in terms of your responses, handling of the
            readings or videos, and/or connections to the assignments? Can
            you describe those patterns?
          \item Reread your assignments and your grading feedback. Were some
            assignments more difficult, more rewarding, or more reflective of
            your interpersonal communication style?
          \item Consider the interpersonal communication skills we studied in
            this course. Choose at least one communication skill or topic from
            each module to answer this question: do you see patterns in terms
            of your responses to those skills or your ability to use them
            effectively?
          \item As related to interpersonal communication, has this course
            enabled you to better become the ``real you'', as Flint encourages
            Sam in the movie clip? Outside of our class, who has most helped
            you become a better interpersonal communicator?
        \end{itemize}
      \item Refrain from responding to the aforementioned questions in
        chronological order, which may result in a final essay that is merely
        a rote reply to each question. Instead, enable the aforementioned
        questions to provide a framework as you begin working on the essay.
      \item Reference your own previous work in the class (Main Discussion
        posts and/or Essays) to support some or all of your responses to the
        aforementioned questions.  Use either APA or MLA Style to format your
        references (minimum of three references).
    \end{enumerate}


  % Essay Body
  % !TEX root=../main.tex

\section{Introduction}
  % Did I Meet Objectives Specified in the Course Syllabus?
  This has been by far the fastest time frame I have ever experienced for a
    college course, and it actually worked out really well for me!
    Unfortunately, due to relationship troubles in addition to travel and other
    personal reasons, I \textit{did} end up being late on an assignment or two.
    This is leaps and bounds better than how well I usually stay attentive
    to my assignments in fifteen week classes, where I will usually do
    incredibly well for the first half of the semester then fall off half way
    through. The time frame of this course is what contributed most to my
    lack of procrastination during the period of this incredibly short semester.

  I believe that I had clearly met the course requirements and objectives in
    terms of defining interpersonal communication, its various skills and terms,
    as well as general communication related vocabulary. In terms of bettering
    myself as an interpersonal communicator, the difference in my thought
    process now when in conversation is enough to prove that, and we will go
    into more detail on this aspect of my experience with the course later in
    this essay.

  The idea that one can ``master'' \textit{any of the hundreds} of interpersonal
    communication skills is insane, as there are always ways to improve. There
    is always a type of person that you and I may have not practiced any sort
    of conversation with, and until you literally converse with \textbf{every
    single human being on the world}, there still exist ways to improve our
    communication!


\section{Patterns Noticed Within My Own Discussion Posts and Responses}
  I regards to my discussion posts, I notice that I almost always tend to open
    each \textit{response} the same way --- with a phrase generally denoting
    ``good work'' and utilizing the name of the author of the original initial
    post \parencite{%
    hellwig_julayne_2020,hellwig_cherylee_2020,hellwig_kristyna_2020%
    }. I believe I could be using this as a crutch, as it is difficult for me
    to come up with an ``opening line'', so to speak, when it comes to
    responding to these discussion posts authored by my classmates.

  I have also noticed that I tend to simply ask the question ``why'' a lot.
    I find that it is the easiest way to (hopefully) continue a conversation.
    We can even see this in my recent discussion response post to Diata Hart
    in the final practice discussion of the course. In that response, perhaps
    my words could have been declared \textit{critical} of Diata's viewpoint
    when it comes to the development of interpersonal communication skills
    \parencite{hellwig_diata_2020}.


\section{An Overview of My Performance With Assigned Essays}
  Essays are the portion of the class in which I felt at my strongest point.
    In \textit{every single essay} up to the point of Module 5, which is being
    written right now, every essay had earned a score above 100\%. I find that
    it is easier to organize my thoughts when I am writing them out over the
    course of a few weeks while using the methods and practices I have developed
    for writing rather than when I am attempting to respond directly to
    other people about their thoughts, rather than just to a prompt.


\section{My Thoughts On Selected Topics Discussed Over The Study Of This Course}
  \subsection{Self-Concept and Perception}
    I feel that it is important to note the ideas of self-concept and
      perception. Self-concept is how one believes that they display themselves
      to the world, and perception is how the outside world views the
      individual. In personal experience, I have found that these two things
      \textit{never match} one another in practice. Especially the generation
      I am a part of where generally the two are complete contrasts to each
      other. Some have a higher opinion of themselves than others really have of
      them, and more often than not people tend to value themselves far lower
      and seek more attention and validation
      \parencite[pp. 586]{stewart_comparing_2010}.

  \subsection{Gender and Cultural Differences}
    Cultural differences between the interpersonal communication and social
      hierarchy are notable, with entire philosophical schools of thought
      dedicated to the differences of how humans live and communicate in
      different cultures (Moral Relativism).

    In terms of the differences between how genders communicate, this is a
      slightly more difficult idea. While there are studies showing the
      slight differences in the expression of empathy and methods utilized
      during conflict resolution in younger sample populations, there is less
      data available as people mature both in their communication skills and
      socially \parencite[pp. 51]{wied_empathy_2007}. Gender can be considered
      more of a spectrum than a binary assignment, which means the way people
      communicate and live about their lives is as unique as fingerprint.

  \subsection{Good Listening Habits}
    Next to the ability to experience and express empathy, good listening
      skills are one of the most important aspects of constructive interpersonal
      communication. Without the ability to retain and act on information being
      provided to you in a welcoming and comforting way, less and less people
      will trust entering a conversation. No one wants to feel interrogated
      or as though they are boring the listener or ``wasting their time''
      throughout a conversation, and it is a major problem generally proceeded
      by arrogant.

  \subsection{The Importance of Nonverbal Communication}
    Nonverbal communication is even more imperative to pay attention to than
      only listening to the words out of a speaker's mouth. It is not as easy
      for people to hide nonverbal communication cues when talking to others,
      because many of them are involuntary. With the requirement of wearing
      masks preventing the ability to see people's mouths when they are
      speaking, we have all grown better at reading each other's \textit{eyes}
      in order to ascertain a person's true feelings.


\section{How This Course Affected My Interpersonal Communication Skills}
  As discussed in my post for Discussion 1 within Module 5 of the course, I
    believe that this course has in fact made me far more self-aware of how I
    interact with those around me \parencite{hellwig_m5d2_2020}.



  % Bibliography
  %% Works Cited
  % \newpage
  % \printbibliography[%
  %   title={References},%
  %   heading={bibintoc},%
  %   notcategory={consulted}%
  % ]


  %% Works Consulted
  \newpage
  \nocite{*}
  \printbibliography[%
    title={Additional References},%
    heading={bibintoc},%
    category={consulted}%
  ]
\end{document}
