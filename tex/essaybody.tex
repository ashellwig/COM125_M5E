% !TEX root=../main.tex

\section{Introduction}
  % Did I Meet Objectives Specified in the Course Syllabus?
  This has been by far the fastest time frame I have ever experienced for a
    college course, and it actually worked out really well for me!
    Unfortunately, due to relationship troubles in addition to travel and other
    personal reasons, I \textit{did} end up being late on an assignment or two.
    This is leaps and bounds better than how well I usually stay attentive
    to my assignments in fifteen week classes, where I will usually do
    incredibly well for the first half of the semester then fall off half way
    through. The time frame of this course is what contributed most to my
    lack of procrastination during the period of this incredibly short semester.

  I believe that I had clearly met the course requirements and objectives in
    terms of defining interpersonal communication, its various skills and terms,
    as well as general communication related vocabulary. In terms of bettering
    myself as an interpersonal communicator, the difference in my thought
    process now when in conversation is enough to prove that, and we will go
    into more detail on this aspect of my experience with the course later in
    this essay.


\section{Patterns Noticed Within My Own Discussion Posts and Responses}
  I regards to my discussion posts, I notice that I almost always tend to open
    each \textit{response} the same way --- with a phrase generally denoting
    ``good work'' and utilizing the name of the author of the original initial
    post \parencite{%
    hellwig_julayne_2020,hellwig_cherylee_2020,hellwig_kristyna_2020%
    }. I believe I could be using this as a crutch, as it is difficult for me
    to come up with an ``opening line'', so to speak, when it comes to
    responding to these discussion posts authored by my classmates.

  I have also noticed that I tend to simply ask the question ``why'' a lot.
    I find that it is the easiest way to (hopefully) continue a conversation.
    We can even see this in my recent discussion response post to Diata Hart
    in the final practice discussion of the course. In that response, perhaps
    my words could have been declared \textit{critical} of Diata's viewpoint
    when it comes to the development of interpersonal communication skills
    \parencite{hellwig_diata_2020}.


\section{An Overview of My Performance With Assigned Essays}
  Essays are the portion of the class in which I felt at my strongest point.
    In \textit{every single essay} up to the point of Module 5, which is being
    written right now, every essay had earned a score above 100\%. I find that
    it is easier to organize my thoughts when I am writing them out over the
    course of a few weeks while using the methods and practices I have developed
    for writing rather than when I am attempting to respond directly to
    other people about their thoughts, rather than just to a prompt.


\section{Patterns Noticed In The Ability To Use Various Skills Discussed}
  Placeholder.

  \subsection{Skill One}
    Placeholder.

  \subsection{Skill Two}
    Placeholder.

  \subsection{Skill Three}
    Placeholder.

  \subsection{Skill Four}
    Placeholder.


\section{How This Course Affected My Interpersonal Communication Skills}
  As discussed in my post for Discussion 1 within Module 5 of the course, I
    believe that this course has in fact made me far more self-aware of how I
    interact with those around me \parencite{hellwig_m5d2_2020}.
